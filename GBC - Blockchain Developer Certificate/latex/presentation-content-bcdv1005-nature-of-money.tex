% (C) Marc Lijour, 2020
% Licensed under a Creative Commons License BY-SA
% https://creativecommons.org/licenses/by-sa/2.5/ca/
% Presentation for the Blockchain Developer Certificate students at George Brown College
% Intro about money

% ======================================================================================================
%                          The Nature of Money
% ======================================================================================================
\section{The Nature of Money}

%\subsection{Definitions}

\frame{ 
	\frametitle{Textbook definition of money}
  \begin{itemize}
    \item a medium of exchange
    \item a unit of account
    \item a store of value
  \end{itemize}
}

\frame{ 
	\frametitle{Money by the pound}
  \begin{itemize}
    \item in medieval europe, money didn't have face value
    \item people would negotiate the value of money
    \item ``pound'' comes from the fact that 240 silver pences weighted about a pound
  \end{itemize}
}

\frame{ 
	\frametitle{Representative money}
  \begin{block}{}
  Alyattes of Lydia, father of Croesus, is famed to be the first monarch to issue coins around 600 BC. The river Pactolus carried Electrum, a gold and silver alloy, and it was easier to trust a stamped value on a coin than measuring the weight of metal for every trade. 
  \end{block}
  \begin{exampleblock}{}
  Later, Charlemagne unified the coinage around the silver penny, before seigneurs and bishops started to mint their own coins, extracting brassage and seignoriage.
  \end{exampleblock}
  \begin{itemize}
    \item Brassage: amount equivalent to the cost of minting
    \item Seignoriage: a tax
  \end{itemize}
}

\frame{
	\frametitle{Fiat money}
  \begin{block}{}
  Fiat money is defined by government regulation, regardless of the value of the underlying commodity.
  \end{block}
  \begin{itemize}
    \item debasement is possible (e.g. using less precious metals, augmenting the quantity of money in circulation)
  \end{itemize}
}

\frame{
  \frametitle{Gresham's law}
  \begin{alertblock}{}
  When two currencies co-exist, bad money drives out good money (e.g. Lebanese pound \& US Dollar in Lebanon).
  \end{alertblock}
}

\frame{
  \frametitle{The Gold standard}
  \begin{itemize}
    \item from the 13th century to the 19th century, paper bills gained adoption
    \item after the Civil War, the US guaranteed the convertibility of bank notes
    \item the gold standard ended after the first world war, when countries such as Germany depreciated their paper money to pay their debt
    \item European countries rekindled with the gold standard during the interwar
    \item those that left the gold standard earlier received economic advantages
    \item after the 2nd world war, the Bretton Woods system has countries peg their currency to the US Dollar
    \item in 1971, Nixon ends the convertibility to gold and the Bretton Woods system
  \end{itemize}
}

\frame{
  \frametitle{Banks}
  \begin{itemize}
    \item central banks progressively appeared in the last two centuries
    \item governments leveraged monetary policy to achieve (or not) social outcomes (e.g. inflation in the 70s and 80s to increase employment)
    \item central banks claim independence from the state, taking over monetary policy
    \item banks are too big to fail (2007 crisis)
    \item central banks are adopting ``unconventional policies'' such as quantitative easing (buying securities on the open market), weakening the reserve requirements (banks must maintain), and lowering interest rates (eventually into negative territory)
    \item Bitcoin appears
  \end{itemize}
}

