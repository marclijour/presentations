% (C) Savoir-faire Linux, 2016 (this document and associated logos and art)
% This document is licensed under a Creative Commons License BY-SA (feel free to use the code, but all rights are reserved for logos and art)
% https://creativecommons.org/licenses/by-sa/2.5/ca/
% Savoir-faire Linux presentation template for LaTeX
% authored by Marc Lijour, December 2016
% This template comes with a first page on a blue background
% Possible improvement in future iterations
% - Test and fix as needed to work on xetex (to use Ubuntu fonts)
% === USAGE===
% Create a file for your LaTeX content (slides, etc), in which you must do the following:
% TODO 1 - set variables defined below
% TODO 2 - include this code by calling: \input{sfl-presentation-template-blue-EN}
% TODO 3 - Start the document as usual and you're in business; just use \begin{document} and don't forget to conclude with \end{document}
% TODO 4 - Use the custom method \SFLcoverpage instead of \titlepage to create your cover page
% Voilà!
%
\documentclass{beamer}
\usepackage{etoolbox}
% Variables
% ---------------------- USER-DEFINED --------------------------------
\ifdef{\SFLtitle}{}{\newcommand{\SFLtitle}{\color{red}Title TBD}}
\ifdef{\SFLlongtitle}{}{\newcommand{\SFLlongtitle}{\color{red}Long title TBD}}
\ifdef{\SFLsubtitle}{}{\newcommand{\SFLsubtitle}{\color{red}Subtitle TBD}}
\ifdef{\SFLauthor}{}{\newcommand{\SFLauthor}{\color{red}Author TBD}}
\ifdef{\SFLdate}{}{\newcommand{\SFLdate}{\color{red}Date TBD}}
\ifdef{\SFLsubject}{}{\newcommand{\SFLsubject}{\color{red}Subject TBD}}
% --------------------------------------------------------------------
\usetheme{Boadilla}
% Set color close to Savoir-faire Linux standard
\definecolor{beamer@blendedblue}{RGB}{86,176,201}
% Cover Page
\title[\SFLtitle] {\SFLlongtitle}
\subtitle{\SFLsubtitle}
\author{\SFLauthor}
\date{\SFLdate}
\subject{\SFLsubject}
\usepackage{tikz}
% -- possible approach through modif of the template (abandonned for now)
%\addtobeamertemplate{title page}{
%    \tikz[remember picture,overlay]
%        \node at ([xshift=0cm,yshift=0cm]current page.center) 
%		 {\includegraphics[width=\paperwidth, height=\paperheight]{./images/sfl-background-blue}};
%}{}
%\setbeamercolor{title page}{fg=white}
%\setbeamercolor{titlelike}{fg=white}
%\setbeamertemplate{navigation symbols}{}
% Try Xetex to use system fonts (pdflatex makes it hard to import a font)
%\usepackage{fontspec}
%\setsansfont{Ubuntu}
%\setmonofont{Ubuntu Mono}
%
%\usepackage[absolute,overlay]{textpos}
%\setlength{\TPHorizModule}{\paperwidth}
%\setlength{\TPVertModule}{\paperheight}
% -- create a custom (command) title page -which has the benefit of not affecting the settings for the rest of the presentation
\newcommand{\SFLcoverpage}{\frame[plain]{
	\tikz[remember picture,overlay] {
        	\node(bkgd) at ([xshift=0cm,yshift=0cm]current page.center) 
			{\includegraphics[width=\paperwidth, height=\paperheight]{../templates/images/sfl-background-blue}};
        	\node(logo) at ([xshift=0cm,yshift=2.5cm]current page.center) 
		 	{\includegraphics[scale=.20]{../templates/images/logo-sfl-blanc-rgb-72dpi}};
        	\node(CC-BY-SA) at ([xshift=5cm,yshift=-3.5cm]current page.center) 
			{\href{https://creativecommons.org/licenses/by-sa/2.5/ca/}{\includegraphics[scale=.4]{../templates/images/CC-BY-SA-403x141}}};
	}
	\tikz[remember picture,overlay] {
        	\node(title) at ([xshift=0cm,yshift=1cm]current page.center) 
			{\Large\color{white}\textbf{{\SFLlongtitle}}};
        	\node(subtitle) at ([xshift=0cm,yshift=.2cm]current page.center) 
			{\small\color{white}\emph{\SFLsubtitle}};
        	\node(author) at ([xshift=0cm,yshift=-2cm]current page.center) 
			{\small\color{white}By~\SFLauthor};
        	\node(date) at ([xshift=0cm,yshift=-2.5cm]current page.center) 
			{\tiny\color{white}\SFLdate};
        	\node(footnote) at ([xshift=0cm,yshift=-4cm]current page.center) 
			{\TINY\color{white}\emph{The registered trademark Linux$^\circledR$ is used pursuant to a sublicense from LMI, the exclusive licensee of Linus Torvalds, owner of the mark on a world-wide basis.}};
    	}
}}
%
% This sets a Savoir-faire Linux logo at the bottom right corner of each page
\logo{
	\includegraphics[scale=.1]{../templates/images/logo-sfl-250.png}
}
\AtBeginSection[]
{
  \begin{frame}
    \frametitle{Table of Contents}
    \tableofcontents[currentsection]
  \end{frame}
}
%\usepackage[format=plain,justification=raggedright,singlelinecheck=false]{caption}
\usepackage[format=plain,justification=justified,singlelinecheck=false]{caption}
\usepackage[utf8]{inputenc}
\usepackage{dirtytalk}
\usepackage{wrapfig}
\usepackage{hyperref}
\usepackage{verbatim}
\usepackage{mathabx}
%\usepackage{MnSymbol}

